%%%%% Tools

\usepackage{comment}
\usepackage{lipsum}
\usepackage{xstring}

%%%%% Document

\usepackage[pdftex]{hyperref}

\makeatletter
\AtBeginDocument{
	\hypersetup{
		pdftitle={\@title},
		pdfauthor={Bastien Hoffmann, Maxime Meurisse},
		pdfkeywords={latex, document},
		pdfproducer={LaTeX document},
		pdfcreator={Document generated with pdflatex}
	}
}
\makeatother

\usepackage{geometry}
\usepackage[parfill]{parskip}

\geometry{paper=a4paper,top=3.5cm,bottom=2.5cm,right=2.5cm,left=2.5cm}

%%%%% Text

\usepackage[utf8]{inputenc}
\usepackage[T1]{fontenc}

\newlength{\mytextsize}
\makeatletter
\setlength{\mytextsize}{\f@size pt}
\makeatother

%%%%% Languages

\usepackage[\languages]{babel}

% english

\addto\captionsenglish{\def\figurename{Figure}}
\addto\captionsenglish{\def\tablename{Table}}

\newcommand{\st}{\text{s.t.}}

\IfStrEq{\languagename}{english}{
	\newcommand{\lgpreamble}{Preamble}
}

% french

\frenchbsetup{StandardLists=true}

\addto\captionsfrench{\def\figurename{Figure}}
\addto\captionsfrench{\def\tablename{Tableau}}
\addto\captionsfrench{\def\proofname{Preuve}}

\newcommand{\tq}{\text{t.q.}}
\newcommand{\cad}{c.-à-d. }
\newcommand{\Cad}{C.-à-d. }

\IfStrEq{\languagename}{french}{
	\newcommand{\lgpreamble}{Préambule}
}

%%%%% Styles

\usepackage[skip=12pt]{caption}
\usepackage{float}
\usepackage{mdframed}
\usepackage{enumitem}
\usepackage{eurosym}
\usepackage{color}
\usepackage[table]{xcolor}

%%%%% Mathematics

\usepackage{amsmath}
\usepackage{amssymb}
\usepackage{amsfonts}
\usepackage{bm}
\usepackage{esint}

\newcommand{\fact}[1]{#1!}
\newcommand{\deriv}{\mathrm{d}}
\DeclareMathOperator{\tr}{tr}

%%%%% SI units

\usepackage[squaren,Gray,cdot]{SIunits}
\usepackage{sistyle}
\SIdecimalsign{,}

%%%%% Chemistry

\usepackage[version=4]{mhchem}

%%%%% Table & Figure

\usepackage{array}
\usepackage{tabularx}
\usepackage{multirow}
\usepackage{multicol}
\newcolumntype{M}[1]{>{\centering\arraybackslash}m{#1}}
%\setlength\extrarowheight{0em}
\renewcommand{\arraystretch}{1.3}

\usepackage{pgfplots}
\usepackage{tikz}
\usetikzlibrary{shapes.geometric, positioning}
\usepackage{graphics}
\usepackage{graphicx}
\pgfplotsset{axis on top, compat = 1.3}
%\setlength{\belowcaptionskip}{0pt}

%%%%%% Theorems and Definitions

\usepackage{amsthm}
\usepackage{thmtools}

\IfStrEq{\languagename}{english}{
	\newcommand{\lgthm}{Theorem}
	\newcommand{\lglem}{Lemma}
	\newcommand{\lgprop}{Proposition}
	\newcommand{\lgdefn}{Definition}
	\newcommand{\lghyp}{Hypothesis}
	\newcommand{\lgquest}{Question}
	\newcommand{\lgansw}{Answer}
	\newcommand{\lgexpl}{Example}
	\newcommand{\lgrmk}{Remark}
	\newcommand{\lgnote}{Note}
	\newcommand{\lgtip}{Tip}
}

\IfStrEq{\languagename}{french}{
	\newcommand{\lgthm}{Théorème}
	\newcommand{\lglem}{Lemme}
	\newcommand{\lgprop}{Proposition}
	\newcommand{\lgdefn}{Définition}
	\newcommand{\lghyp}{Hypothèse}
	\newcommand{\lgquest}{Question}
	\newcommand{\lgansw}{Réponse}
	\newcommand{\lgexpl}{Exemple}
	\newcommand{\lgrmk}{Remarque}
	\newcommand{\lgnote}{Note}
	\newcommand{\lgtip}{Conseil}
}

\theoremstyle{plain}
\newtheorem{thm}{\lgthm}[section]
\newtheorem{lem}{\lglem}[section]
\newtheorem{prop}{\lgprop}[section]

\theoremstyle{definition}
\newtheorem{defn}{\lgdefn}[section]
\newtheorem{hyp}{\lghyp}[section]
\newtheorem{quest}{\lgquest}[]

\declaretheorem[
name=\lgansw,
qed={\lower-0.3ex\hbox{$\triangle$}},
within=quest
]{answ}

\declaretheorem[
name=\lgexpl,
qed={\lower-0.3ex\hbox{$\triangle$}},
within=section
]{expl}

\theoremstyle{remark}
\newtheorem*{rmk}{\lgrmk}
\newtheorem*{note}{\lgnote}
\newtheorem*{tip}{\lgtip}

\begingroup
\makeatletter
\@for\theoremstyle:=definition,remark,plain\do{%
	\expandafter\g@addto@macro\csname th@\theoremstyle\endcsname{%
		\addtolength\thm@preskip\parskip
	}%
}
\endgroup

%%%% Code

\usepackage{listings}

\definecolor{jblue}{rgb}{0.13,0.13,1}
\definecolor{jgreen}{rgb}{0,0.5,0}
\definecolor{jred}{rgb}{0.9,0,0}

\lstdefinestyle{MyJava}{
	language=Java,
	%%%%%%
	showstringspaces=false,
	extendedchars=true,
	tabsize=4,
	columns=fixed,
	%%%%%%
	breaklines=true,
	breakatwhitespace=true,
	prebreak=\space,
	%%%%%%
	basicstyle=\footnotesize\ttfamily,
	keywordstyle=\color{jblue},
	commentstyle=\color{jgreen},
	stringstyle=\color{jred},
	%%%%%%
	numbersep=0.2\mytextsize,
	numbers=left,
	numberstyle={\ttfamily\footnotesize},
	%%%%%%
	frame=single,
	rulecolor=\color{black},
	framexleftmargin=2\mytextsize,
	xleftmargin=2\mytextsize,
	captionpos=b
}

\lstdefinestyle{MyMatLab}{
	language=Matlab,
	%%%%%%
	showstringspaces=false,
	extendedchars=true,
	tabsize=4,
	columns=fixed,
	%%%%%%
	breaklines=true,
	breakatwhitespace=true,
	prebreak=\space,
	%%%%%%
	basicstyle=\footnotesize\fontfamily{pcr},
	keywordstyle=\color[rgb]{0,0,1},
	commentstyle=\itshape\color{green!40!black},
	stringstyle=\color[rgb]{.627,.126,.941},
	%%%%%%
	numbersep=0.5\mytextsize,
	numbers=left,
	numberstyle={\lstbasicfont\footnotesize},
	%%%%%%
	frame=single,
	rulecolor=\color{black},
	framexleftmargin=2\mytextsize,
	xleftmargin=2\mytextsize,
	captionpos=b
}

\lstdefinestyle{MyVHDL}{
	language=VHDL,
	%%%%%%
	showstringspaces=false,
	extendedchars=true,
	tabsize=4,
	columns=fixed,
	%%%%%%
	breaklines=true,
	breakatwhitespace=true,
	prebreak=\space,
	%%%%%%
	basicstyle=\footnotesize\ttfamily,
	keywordstyle=\color{blue!100!black!80},
	commentstyle=\itshape\color{green!90!black!90},
	stringstyle=\color[rgb]{.627,.126,.941},
	%%%%%%
	numbersep=0.5\mytextsize,
	numbers=left,
	numberstyle={\footnotesize},
	%%%%%%
	frame=single,
	rulecolor=\color{black},
	framexleftmargin=2\mytextsize,
	xleftmargin=2\mytextsize,
	captionpos=b
}

\usepackage{clrscode3e}

%%%% Others

\renewcommand{\qedsymbol}{$\blacksquare$}

%%%%%%%%%%%%%%%%%%%